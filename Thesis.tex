% !Mode:: "TeX:UTF-8"   %%winedt 以utf8编码打开
% Can only be compiled with xelatex+bibtex

%双行标题
\documentclass[twoside,longtitle]{LZUthesis}
%单行标题
% \documentclass[twoside]{LZUthesis}

\usepackage{float}
\usepackage{fancybox}
\usepackage{calc}
\usepackage{mathdots}
\usepackage{graphicx}
\usepackage{listings}

%声明图片后缀名
\DeclareGraphicsExtensions{.pdf,.eps}

\makeatletter

% 设置图形文件的搜索路径
\graphicspath{{figures/}}

% 小节标题靠左对齐
\CTEXsetup[format+={\flushleft}]{section}

% 取消链接的颜色(黑白打印时启用)
%\hypersetup{colorlinks=false}

%使文档居中,打印时应注释掉
\evensidemargin 0.93 true cm
\setlength{\hoffset}{-0.3 cm}

%允许equationarray分页换行
\allowdisplaybreaks

%使字体清晰,且透明位图不会使页面文字粗细不一
\usepackage{eso-pic}
\AddToShipoutPicture{%
\special{pdf: put @thispage <</Group << /S /Transparency /I true /CS /DeviceRGB>> >>}%
}

%页面背景色
%\definecolor{yellow}{rgb}{0.99,0.99,0.70}
%\pagecolor{yellow}

%浮动项超链接正确跳转
\usepackage[all]{hypcap}

\makeatother


\begin{document}
	
% 学校代码
\schoolcode{111}

%分类号;一般不要求学生填写
\classification{11}

%密级;申请保密后需填写
\confidential{11}

%中文标题
\title{兰州大学研究生学位论文}

%中文标题第二行,题目较短时删除之,并去除文档类选项 longtitle
\titleadd{\LaTeX{}模板简介}

%英文标题
\englishtitle{Introduction to the \LaTeX{} Template of}

%英文题目第二行,题目较短时删除之,并去除文档类选项 longtitle
\englishtitleadd{Lanzhou University Thesis}

%作者汉语姓名
\author{XXX}

%专业
\major{应用统计}

%学位
\degree{学历教育}

%研究方向
\direction{XXXX}

%导师姓名+职称
\advisor{某某某~教授}

% 合作者
\collaborator{}

%论文工作起止年月
\datebeginAndEnd{2019年4月至2020年3月}

%论文提交日期
% \submitdate{}

%答辩日期
\defenddate{}

%学位授予日期
% \degreedate{}

%左侧页眉
\lzuthesis{兰州大学硕士学位论文}

%生成封面
\maketitle

%生成声明页
\makestatement

%前文-罗马页码
\frontmatter\pagenumbering{Roman}

%中文摘要
\begin{abstract}
本文是兰州大学学位论文的\LaTeX{} (包含LyX)模板及其使用说明。除了介绍文档类LZUthesis的用法外,本文还是一份简要的学位论文写作指南。
\end{abstract}

%中文关键词
\keywords{兰州大学,学位论文,\LaTeX{}模板,LyX模板}

%英文摘要
\begin{englishabstract}
This paper is a \LaTeX{} (include LyX) thesis template of Lanzhou University. Besides the usage of the LaTeX document class LZUthesis, a brief guideline for writing the thesis is also included.
\end{englishabstract}

%英文关键词
\englishkeywords{Lanzhou University, Thesis, \LaTeX{} Template, LyX Template}

%自动生成章节目录以及pdf书签
\tableofcontents{}


%正文部分-数字页码
\mainmatter

%正文页眉页脚样式
\pagestyle{lzu}


\chapter{引言\label{chap:intro}}

虽然兰州大学研究生院制定了\href{http://ge.lzu.edu.cn/degree/xwsq/lwgf/201103/1186.htm}{学位论文撰写要求},但只提供了\href{http://ge.lzu.edu.cn/degree/xwsq/lwgf/201103/1187.htm}{~Word 模板}。广大理科生更加青睐的\LaTeX{}却没有标准模板。\href{http://blog.sciencenet.cn/home.php?mod=space&uid=117412&do=blog&id=512804}{LZUthesis~} 宏包是赵振华以中科院学位论文宏包\href{http://www.ctex.org/PackageCASthesis}{~CASthesis~}作为基础模板,根据兰州大学研究生院学位论文撰写要求修改的。作者对LZUthesis宏包进行了进一步修改(主要变更见附录~\ref{chap:changelog}~及\href{https://github.com/mosesnow/LZUthesis/commits/master}{~Github~}),使之更加符合兰大学位论文的撰写要求,并制作了相应的LyX模板。

宏包底层使用的是Xe\LaTeX{},而不是CJK方案,原因在于:
\begin{enumerate}
\item Xe\LaTeX{}支持系统字体,对中文友好。
\item Xe\LaTeX{}支持unicode,可编译各国文字。
\item Xe\LaTeX{}支持常见的图片格式,如:pdf、eps、jpg、png 等。
\end{enumerate}


模板使用LyX的优点是所见即所得,比较直观,缺点是\TeX{}中一些复杂功能难以实现。另外LyX可以导出标准的\TeX{}代码(参见第~\ref{sec:export}~节),方便转入\TeX{}编辑器继续撰写。

青睐\TeX{}模板的用户,直接编辑模板根目录下的Thesis.tex即可。注意它不是Thesis.lyx直接导出的,如果要使用此模板,不要让LyX导出\TeX{}时覆盖此文件。


\chapter{系统配置}

\texttt{LZUthesis}宏包使用了宏包amsmath、amsthm、amsfonts、amssymb、bm、ctex 和hyperref。目前大多数的\TeX{}系统中都包含有这些宏包。但是中文字体在各个系统中却不尽相同:Windows 下使用C\TeX{} 套装一般不需要额外配置字体,但Linux 和Mac一般情况下需要设定中文字体,中文字体名称可通过命令~\lstinline!sudo fc-list :lang=zh-cn!~ 查看,具体配置见本章内容。请注意字体设定可能要根据你自己系统的中文字体做相应调整。


\section{Windows配置}
\begin{itemize}
\item 安装C\TeX{}套装,最新的版本可以从\href{http://www.ctex.org}{~www.ctex.org} 网站下载。
\item 安装完成后,把所有宏包升级到最新(开始菜单--CTEX--Update(Admin))。
\item 安装LyX,最新的LyX安装包可以从\href{http://www.lyx.org/Download}{~www.lyx.org/Download} 下载。
\end{itemize}

\section{Linux配置}

软件安装以ubuntu为例,其他发行版应根据具体情况安装。
\begin{itemize}
\item 安装texlive以及texlive-lang-cjk。


\doublebox{\begin{minipage}[t]{0.86\textwidth}%
\begin{lstlisting}[language=bash,breaklines=true]
sudo apt-get install texlive texlive-xetex
sudo apt-get install texlive-lang-cjk
\end{lstlisting}
%
\end{minipage}}

\item 安装LyX。


\doublebox{\begin{minipage}[t]{0.86\textwidth}%
\begin{lstlisting}[language=bash,breaklines=true]
sudo apt-get install lyx
\end{lstlisting}
%
\end{minipage}}

\item 中文字体设置

\begin{enumerate}
\item 文档类选项(文档--首选项--文档类)中添加选项{[}nofonts{]}
\item 导言区(文档--首选项--\LaTeX{}导言区)中添加(Forked from\href{https://bitbucket.org/kevinyounglzu/latextemplateoflzuthesis}{~Kevin Young's tempalte})


\tiny
\begin{center}
\hrule


\begin{verbatim}


\setCJKmainfont[BoldFont=文泉驿微米黑,ItalicFont=文泉驿正黑] {文泉驿正黑} \setCJKsansfont{文泉驿微米黑}
\setCJKmonofont{文泉驿等宽正黑}
\setCJKfamilyfont{zhsong}{文泉驿正黑}
\setCJKfamilyfont{zhhei}{文泉驿微米黑}
\setCJKfamilyfont{zhfs}{文泉驿等宽微米黑}
\setCJKfamilyfont{zhkai}{文泉驿等宽正黑}
\newcommand*{\songti}{\CJKfamily{zhsong}} % 宋体
\newcommand*{\heiti}{\CJKfamily{zhhei}}   % 黑体
\newcommand*{\kaishu}{\CJKfamily{zhkai}}  % 楷书
\newcommand*{\fangsong}{\CJKfamily{zhfs}} % 仿宋


\end{verbatim}


\hrule
\end{center}
\normalsize

\end{enumerate}
\end{itemize}

\section{Mac配置}
\begin{itemize}
\item 安装Mac\TeX{},最新安装文件可以从\href{https://tug.org/mactex/}{~https://tug.org/mactex/}
网站下载。
\item 安装LyX,最新的LyX安装包可以从\href{http://www.lyx.org/Download}{~www.lyx.org/Download} 下载。
\item 中文字体设置

\begin{enumerate}
\item 文档类选项(文档--首选项--文档类)中添加选项{[}nofonts{]}
\item 导言区(文档--首选项--\LaTeX{}导言区)中添加(Forked from\href{https://bitbucket.org/kevinyounglzu/latextemplateoflzuthesis}{~Kevin Young's tempalte})


\tiny
\begin{center}
\hrule


\begin{verbatim}


\setCJKmainfont[BoldFont=Heiti SC Light,ItalicFont=Kaiti SC]   {Songti SC Light}
\setCJKsansfont{Heiti SC Light} \setCJKmonofont{Songti SC Light}
\setCJKfamilyfont{zhsong}{Songti SC Light}
\setCJKfamilyfont{zhhei}{Heiti SC Light}
\setCJKfamilyfont{zhfs}{STFangsong}
\setCJKfamilyfont{zhkai}{Kaiti SC}
\newcommand*{\songti}{\CJKfamily{zhsong}} % 宋体
\newcommand*{\heiti}{\CJKfamily{zhhei}}   % 黑体
\newcommand*{\kaishu}{\CJKfamily{zhkai}}  % 楷书
\newcommand*{\fangsong}{\CJKfamily{zhfs}} % 仿宋


\end{verbatim}


\hrule
\end{center}
\normalsize

\end{enumerate}
\end{itemize}




\chapter{模版使用}


\section{模板文件结构\label{sec:files}}

整个模板根目录的文件列表如下:

\begin{tabular}{|l|l|l|l|}
\hline
LZUthesis.cls & ---LZUthesis宏包 & \textcolor{red}{{*}} & \textcolor{magenta}{\$}\tabularnewline
\hline
LZUthesis.cfg & ---LZUthesis宏包配置文件 & \textcolor{red}{{*}} & \textcolor{magenta}{\$}\\
\hline
LZUthesis.layout & ---LZUthesis的LyX布局文件 & \textcolor{red}{{*}} & \textcolor{magenta}{\$}\\
\hline
CASmthesis.cls & ---CASthesis宏包 & \textcolor{red}{{*}} & \textcolor{magenta}{\$}\\
\hline
CASmthesis.cfg & ---CASthesis宏包配置文件 & \textcolor{red}{{*}} & \textcolor{magenta}{\$}\\
\hline
lzubib.bst & ---引文样式文件 & \textcolor{red}{{*}} & \textcolor{magenta}{\$}\\
\hline
bib/thesis.bib & ---bib数据库 & \textcolor{red}{{*}} & \textcolor{magenta}{\$}\\
\hline
figures/lzu.pdf & ---兰州大学校名标准字 & \textcolor{red}{{*}} & \textcolor{magenta}{\$}\\
\hline
Thesis.lyx & ---LyX模板主文件 & \textcolor{red}{{*}} & -\\
\hline
Envelope.lyx & ---A3封面LyX模板 & \textcolor{red}{{*}} & -\\
\hline
spine.lyx & ---书脊LyX模板 & \textcolor{red}{{*}} & -\\
\hline
Thesis.tex & ---\TeX{}模板主文件 & - & \textcolor{magenta}{\$}\\
\hline
Makefile & ---Makefile文件 & - & -\\
\hline
gbt7714-2005.bst & ---符合国标gbt7714-2005的引文样式文件 & - & -\\
\hline
\end{tabular}

注:\textcolor{red}{{*}}表示LyX模板必须的文件,\textcolor{magenta}{\$} 表示\LaTeX{}模板必须的文件。


\section{基本设置}
\begin{enumerate}
\item 模版默认使用的文档类选项(文档--首选项--文档类--自定义)是twoside,~longtitle,即双面,长标题。

\begin{enumerate}
\item 双面:此时页码、页边距以及章开始页等会对奇偶页区别设置,可更改为oneside。
\item 长标题:若标题较短时应去除选项longtitle,并删除中英文标题的第二行。
\end{enumerate}
\item 图片搜索路径默认设置为模板根目录下的figures/。
\item bib数据库默认设置为模板根目录下的bib/thesis.bib。
\end{enumerate}

\section{示例}

论文中最常使用的一些功能在本节中给出示例,其他LyX功能的使用详见帮助-- 用户手册,\TeX{}使用则可参考The \TeX{}book\cite{Knut04-T}。


\subsection{公式}

自旋玻色模型的哈密顿量为
\begin{equation}
\hat{H}=\frac{\epsilon}{2}\hat{\sigma}_{z}-\frac{\Delta}{2}\hat{\sigma}_{x}+\sum_{k}\omega_{k}\hat{b}_{k}^{\dagger}\hat{b}_{k}+\sum_{k}\frac{g_{k}}{2}\hat{\sigma}_{z}(\hat{b}_{k}+\hat{b}_{k}^{\dagger})\label{eq:sbm}
\end{equation}



\subsection{表格}

前五个希腊字母
\begin{table}[H]
\begin{centering}
\begin{tabular}{|c|c|c|c|c|}
\hline
Alpha & Beta & Gamma & Delta & Theta\\
\hline
$\alpha$ & $\beta$ & $\gamma$ & $\delta$ & $\theta$\\
\hline
$A$ & $B$ & $\Gamma$ & $\Delta$ & $\Theta$\\
\hline
\end{tabular}
\par\end{centering}

\protect\caption{希腊字母表\label{tab:Greek}}


\end{table}



\subsection{图形}
兰州大学校名标准字
\begin{figure}[H]
\begin{centering}
\includegraphics[width=0.618\textwidth]{figures/lzu}
\par\end{centering}

\protect\caption{兰州大学校名标准字\label{fig:lzu}}
\end{figure}



\subsection{引用}


\subsubsection{交叉引用}

对所有需要引用的公式、表格、图形,执行插入--标签后,即可使用插入-- 交叉引用自动产生引用。
\begin{itemize}
\item 哈密顿量见方程~(\ref{eq:sbm})。
\item 希腊字母表见表~\ref{tab:Greek}。
\item 校名标准字如图~\ref{fig:lzu}。
\end{itemize}

\subsubsection{文献引用}

将引文的bib数据库(默认文件名为thesis.bib)放入模板根目录下的bib 文件夹,即可通过插入--文献引用自动产生引文。
\begin{itemize}
\item Journal:An article\cite{Liu13PRA-A}。
\item Book:C\TeX{} FAQ\cite{Wu05-C}。
\end{itemize}

\subsection{Makefile}

Makefile(Forked from\href{https://bitbucket.org/kevinyounglzu/latextemplateoflzuthesis}{~Kevin Young's tempalte})仅限\LaTeX{}模板编译使用,适合Linux和Mac用户使用。
\begin{enumerate}
\item make:只编译源文件,快速调试用。
\item make ref:包含引用的完整编译。
\item make clean:清除临时文件。
\end{enumerate}

\section{注意事项}
\begin{enumerate}
\item 模板目录下的必须文件(见第~\ref{sec:files}~节)注意不要删除。
\item 请尽量升级\TeX{}宏包到最新,一些旧包会使排版稍微不正常。截止到\today ,此模板在最新宏包下工作正常。
\item Bibkey中不能有中文字符。
\item LyX中必须用空格时,尽量用\textasciitilde{}或(插入--格式-- 强制间距~) 代替,不用空格的原因是:LyX不会中文断句,使用空格将使得LyX编辑界面混乱。
\item Hyperref已经在模板中声明过,如果在LyX中执行插入--超链接会造成冲突(直接编辑\TeX{}代码无此问题),这时只能通过插入\TeX{} 代码的方式使用,比如:\lstinline!\href{https://www.google.com}{Google}!。
\item LyX中,enumerate环境结束后,必须先添加一个分割线环境才能换行,具体参见第\ref{chap:intro}章中列表结束后的蓝色分割线(蓝色分割线仅现于LyX)。
\end{enumerate}

\section{论文打印}


\subsection{文档设置}
\begin{enumerate}
\item 按需选取twoside或者oneside的选项。
\item 文档类notypeinfo选项去除typeinfo。
\item 启用导言区(文档--首选项--\LaTeX{}导言区)的~\lstinline!\hypersetup{colorlinks=false}!以取消链接的颜色。
\item 预留左侧装订线:注释掉导言区中如下两行


\doublebox{\begin{minipage}[t]{0.86\textwidth}%
\begin{lstlisting}[language=bash,breaklines=true]
\evensidemargin 0.93 true cm
\setlength{\hoffset}{-0.3 cm}
\end{lstlisting}
%
\end{minipage}}

\end{enumerate}

\subsection{书脊生成}
\begin{enumerate}
\item spine.lyx生成竖排的书脊文字。
\item Envelope.lyx生成含书脊的封皮,纸张大小A3+。
\item 若中文标题中含有英文,需要将导言区(文档--首选项--\LaTeX{} 导言区)中的\lstinline!\shuji! 替换为如下样式:\footnotesize
\begin{lstlisting}
\shuji[某\hspace{0.3em}\raisebox{-5pt}{English}\hspace{-0.1em}~某]
\end{lstlisting}
\normalsize 本文标题中的\LaTeX{}即是如此处理,对于全中文标题的情况可做上面的相反替换。
\end{enumerate}

\chapter{进阶功能}


\section{导出\protect\TeX{}\label{sec:export}}

文件-导出-\LaTeX{}(Xe\TeX{}),可导出为tex文件进行进一步编辑,该tex 文件的编译顺序为: Xe\LaTeX{}--Bib\TeX{}--Xe\LaTeX{}--Xe\LaTeX{}。 注意LyX 默认会自动将图片转换为pdf格式,造成一些图片有两个版本,不处理也不会有影响。若想处理,建议保留pdf版本的,因为Xe\LaTeX{}处理pdf图片时,编译速度快;若选择保留其他格式,应在tex文件正文前声明图片的后缀名,比如:\lstinline!\DeclareGraphicsExtensions{.pdf,.eps}!


\section{导入Bibitem到正文}

如果需要在没有bib文件的情况下也可编译,则首先把导出的tex文件完整编译一次,再将生成的bbl文件内容拷贝到tex正文中替换掉如下两行即可

\doublebox{\begin{minipage}[t]{0.86\textwidth}%
\begin{lstlisting}[language=bash,breaklines=true]
\bibliographystyle{lzubib}
\bibliography{bib/thesis}
\end{lstlisting}
%
\end{minipage}}


\section[修复stemV问题]{修复stemV问题\footnote{此问题已在修复透明位图导致文字粗细不一的问题时同时得到解决。}}

由于中文字体某些参数设置不当,\LaTeX{}生成的pdf在Adobe阅读器下中文字体(特别是宋体与仿宋)会稍显模糊,可通过修改stemV值改善。
\begin{enumerate}
\item 下载并安装\href{https://www.pdflabs.com/tools/pdftk-the-pdf-toolkit/}
{~pdftk free}
\item cmd中切换到pdf所在目录,执行


\doublebox{\begin{minipage}[t]{0.86\textwidth}%
\begin{lstlisting}[language=bash,breaklines=true]
pdftk in.pdf output out.pdf uncompress
\end{lstlisting}
%
\end{minipage}}

\item 用文本编辑器编辑out.pdf,查找并修改simsun与fangsong的stemV 值为50,保存。
\item 执行以下命令,得到final.pdf


\doublebox{\begin{minipage}[t]{0.86\textwidth}%
\begin{lstlisting}[language=bash,breaklines=true]
pdftk out.pdf output final.pdf compress
\end{lstlisting}
%
\end{minipage}}

\end{enumerate}

\section{自动生成成果页}
\begin{enumerate}
\item 此方法仅适用于\LaTeX{}模板。
\item 导言中添加:


\doublebox{\begin{minipage}[t]{0.86\textwidth}%
\begin{lstlisting}[language=bash,breaklines=true]
%%自动生成发表文章目录
\usepackage[resetlabels]{multibib}
%把参考文献加入目录中,但排除目录本身
\usepackage[nottoc]{tocbibind}
\newcites{phd}{在学期间的研究成果}
\end{lstlisting}
%
\end{minipage}}

\item 在参考文献列表后添加


\doublebox{\begin{minipage}[t]{0.86\textwidth}%
\begin{lstlisting}[language=bash,breaklines=true]
%%自动生成发表文章目录
%此处插入所有发表文章的bibkey
\nocitephd{Liu13PRA-A}
%定义引文样式为lzubib
\bibliographystylephd{lzubib}
%发表文章的bib库
\bibliographyphd{bib/thesis}
\end{lstlisting}
%
\end{minipage}}

\item 此种情况使用makefile不能完整编译,但WinEdt可以完整编译。执行顺序为:


Xe\LaTeX{}--Bib\TeX{}--Xe\LaTeX{}--Xe\LaTeX{}。


命令行下手动编译时需要额外执行\lstinline!BibTeX phd.aux!。

\end{enumerate}



%附录部分
\appendix

%自动生成参考文献列表,需要模板根目录下有:bib/thesis.bib
\bibliographystyle{lzubib}
\bibliography{bib/thesis}


%发表文章目录
\begin{publications}{99}
\item Ji-Hong Zhang, \textbf{Ling-Yun Wu}, and Xiang-Sun Zhang, \textit{Reconstruction of DNA sequencing by hybridization}, \href{http://link.aps.org/doi/10.1103/PhysRevA.87.052139}{Bioinformatics, \textbf{19}(1): 14-21.} (2013).
\item \textbf{吴凌云}, \href{http://www.ctex.org/CTeXFAQ}{\textit{CTEX FAQ (常见问题集)}}, (2004).
\end{publications}


%致谢
\begin{thanks}
Ajouter de la vie aux jours lorsqu'on ne peut plus ajouter de jours
à la vie.

\begin{flushright}
---Jean Bernard
\par\end{flushright}
\end{thanks}


\chapter{v1.0主要变更\label{chap:changelog}}
\begin{enumerate}
\item 修复目录中摘要和abstract超链接的错误
\item 修复长标题在摘要部分只显示第一行
\item 修复中英文摘要第一行不缩进
\item 修复页眉上不显示完整标题
\item 页眉改为双线
\item 替换lzu.eps为清晰版本
\item 添加lyx模板
\item 修正行间距
\item 设置Twoside时页码在装订线右侧,oneside时居中
\item 调整封面、声明页布局
\item 修正章标题为三号字
\item 改变图表的编号格式为2-1
\item 修正致谢页的字体及行距
\item 修正一些中文字段,如摘要、Key words、在学期间的研究成果等
\item 修改小节标题为不缩进
\item 添加A3封面,书脊
\item 修改bst样式文件
\item 调小装订线距离为0.6cm
\item 致谢、成果偶数页留白
\item 单独定义makestatement
\item 合并cls文件,longtitle作为类选项进入
\item 更改各处全标题的调用
\item 成果页格式调整为与参考文献相同
\item 摘要重定义为chapter,间距微调
\item chapter标题上下间距微调
\item 去除摘要、Abstract、目录的页眉
\end{enumerate}

\end{document}
